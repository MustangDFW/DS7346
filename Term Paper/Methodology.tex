Methodology

Machine learning techniques along with Deep Learning are widely used in application for various exo-planet scientific research. One thing that is quite uncertain is the number of actual planets that exist outside our own solar system. There are various techniques that can be utilized to help us with this detection. Before delving into the techniques that can be used for prediction, data management needs to be touched upon. When it comes to any analysis, data is required. However, where the data is stored can be a huge concern. Traditional databases go a long way, but in order to do efficient and thorough work, it is important to store this information in a secure and accessible way. Cloud technology makes hosting this type of data easy. By utilizing Amazon Web Services (AWS) S3 and RDS (Relational Database Service) services this task can be accomplished. The benefits of having this storage and data management power, is scalability as well as quality [16]. Especially when working with data that is ever growing in the "Planet/Space" field, scalability is huge. 

As touched upon, Amazon's S3 service helps with storage of clean and robust data. The option to expand storage as well as version control will alleviate any burden of mismatched data. Traditional storage services like DropBox are great for storing data but not really helpful in terms of version control. This can be done in S3 services. Performance is also an important factor when it comes to manipulation. Amazon S3 is ideal when it comes to fast delivery times with the options of image and video rendering [7]. This might be fruitful when more renders of exo-planets can be collected. On top of its power to store data, analysis can be done with in-place querying functionality and there are many third party tools integrated that can be used [7]. Overall, AWS S3 will be the most leveraged service for its all around capabilities.

AWS also offers RDS services that can be further used for storage and querying. Similar to S3, Amazon RDS has many of the same characteristics but is tailored towards the database realm. Some prime attributes include Scalability, Availability, Vertical Scalability, Horizontal Scalability, and Performance [8]. How effectively this service can be utilized in terms of this project will be discussed further, as the primary winner will be S3 services. 

Additionally throughout this paper, it is also essential to create a models that can detect whether or not life can be accommodated in these various exo-planets. One of the main Machine Learning methods that can be utilized that is beneficial to this topic is classification. Classification models are one of the key supervised techniques that can be leveraged in exo-planet detection. Based on our research from various literature the following methods would be best suited for this type of work. 

In Cameron et. al (2018) there were numerous classification techniques that were used for exo-planet detection. The following classification techniques were used: Support Vectors, Linear Support Vector, K-Nearest Neighbors, Random Forest, and finally Logistic Regression. Ideally, these supervised techniques will be the easiest and more suitable to be run. This can be due to the fact that these techniques have been in practice and are some staple techniques. The research and performance will be discussed more detailed in the ML milestones section later on this paper.








[15].	Schanche, N., Cameron, A. C., Hébrard, G., Nielsen, L., Triaud, A. H. M. J., Almenara, J. 	M., Alsubai, K. A., Anderson, D. R., Armstrong, D. J., Barros, S. C. C., Bouchy, 	F., Boumis, P., Brown, D. J. A., Faedi, F., Hay, K., Hebb, L., Kiefer, F., Mancini, 	L., Maxted, P. F. L., … Wheatley, P. J. (2018). Machine-learning approaches to 	exoplanet transit detection and candidate validation in wide-field ground-based 	surveys. Monthly Notices of the Royal Astronomical Society, 483(4), 5534–5547. 	https://doi.org/10.1093/mnras/sty3146

[16] Ohri, A. (2021, March 03). Cloud data MANAGEMENT: An interesting Overview (2021). Retrieved March 05, 2021, from https://www.jigsawacademy.com/blogs/cloud-computing/cloud-data-management/

[17] Amazon S3 - the benefits and how to get started. (2019, August 06). Retrieved March 05, 2021, from https://n2ws.com/blog/aws-cloud/amazon-s3-tutorial

[18] Blaisdell, R. (2015, March 24). Benefits for using aws rds. Retrieved March 05, 2021, from https://rickscloud.com/benefits-for-using-aws-rds/