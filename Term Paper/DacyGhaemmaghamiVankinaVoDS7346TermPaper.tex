%%%%%%%%%%%%%%%%%%%%%%%%%%%%%%%%%%%%%%%%%%%%%%%%%%%%%%%%%%%%%%%%%%%%%%%%%%%%%%%%
%2345678901234567890123456789012345678901234567890123456789012345678901234567890
%        1         2         3         4         5         6         7         8

\documentclass[letterpaper, 10 pt, conference]{ieeeconf}  % Comment this line out
                                                          % if you need a4paper
%\documentclass[a4paper, 10pt, conference]{ieeeconf}      % Use this line for a4
                                                          % paper

\IEEEoverridecommandlockouts                              % This command is only
                                                          % needed if you want to
                                                          % use the \thanks command
\overrideIEEEmargins
% See the \addtolength command later in the file to balance the column lengths
% on the last page of the document

\usepackage[utf8]{inputenc}
\usepackage[T1]{fontenc}
\usepackage{cite}
\usepackage{hyperref}
\usepackage{graphicx}

% The following packages can be found on http:\\www.ctan.org
%\usepackage{graphics} % for pdf, bitmapped graphics files
%\usepackage{epsfig} % for postscript graphics files
%\usepackage{mathptmx} % assumes new font selection scheme installed
%\usepackage{mathptmx} % assumes new font selection scheme installed
%\usepackage{amsmath} % assumes amsmath package installed
%\usepackage{amssymb}  % assumes amsmath package installed

\title{\LARGE \bf
Detection and Analysis of
Exoplanets using Machine Learning
Techniques
}

%\author{ \parbox{3 in}{\centering Lance Dacy*
%         \thanks{*Use the $\backslash$thanks command to put information here}\\
%         Faculty of Electrical Engineering, Mathematics and Computer Science\\
%         University of Twente\\
%         7500 AE Enschede, The Netherlands\\
%         {\tt\small h.kwakernaak@autsubmit.com}}
%         \hspace*{ 0.5 in}
%         \parbox{3 in}{ \centering Thomas Karba**
%         \thanks{**The footnote marks may be inserted manually}\\
%        Department of Electrical Engineering \\
%         Wright State University\\
%         Dayton, OH 45435, USA\\
%         {\tt\small pmisra@cs.wright.edu}}
%}

\author{Lance Dacy$^{1}$, Aurian Ghaemmaghami$^{2}$, Aniketh Vankina$^{3}$, and Jamie Vo$^{4}$% <-this % stops a space
}

\begin{document}

\maketitle
\thispagestyle{empty}
\pagestyle{empty}

%%%%%%%%%%%%%%%%%%%%%%%%%%%%%%%%%%%%%%%%%%%%%%%%%%%%%%%%%%%%%%%%%%%%%%%%%%%%%%%%

\begin{abstract}

The Earth's population continues to grow at a steady rate. The natural resources on Earth are in limited supply. The need to find other worlds that could contain life becomes more focused and intense in the past few years. Scientists desire to find exoplanets that have the features similar to Earth for sustaining life, not only for curiosity sake, but for potential new places for humans to thrive. The data collected over the centuries is so large that filtering the data becomes problematic. With new technologies that use a machine's processing power, data science methods could help scientist evaluate this data and its patterns to narrow down exoplanets that could sustain life as we know it. The data needed to model these exoplanets need to be easily consumed; this project will focus on the viability of deploying cloud data stores specifically to aid in exoplanet modeling.

\end{abstract}


%%%%%%%%%%%%%%%%%%%%%%%%%%%%%%%%%%%%%%%%%%%%%%%%%%%%%%%%%%%%%%%%%%%%%%%%%%%%%%%%
\section{Introduction}

Are we alone in the Universe? That is an age-old question that humans having been striving to answer since we first identified that our planet is simply a small part of a larger whole. Earth appears to be one of the planets in the whole Universe that has just the right ingredients to host life. Or is it? The science continues to explore that question "Are we alone?" and have developed a systematic approach of the course of centuries to help compile data that might just answer that question. 

We live in a technology age where it is feasible to gather mounds of data about our Universe. Even better, we live in a technology age where computers can assist in stitching that data together to help find the patters that point us in the right direction for an answer. Our Data Science Team joins the ranks of numerous scientist to help analyze the  myriad data gathered from the Universe to find the building blocks essential to host life as we know it. Given the quest for life as we know it, scientist narrow down the building blocks to a simple acronym called CHNOPS. CHNOPS stands for Carbon, Hydrogen, Oxygen, Phosphorus, and Sulfur. These are the base elements believed to provide the building blocks for living organisms. Finding a place in the Universe that contains these elements is much like finding a needle in a haystack.

Aside from the fact that the human race has a quest to find others in the Universe, there is also another need that needs to be sustained. We will eventually consume all the natural resources on this precious Earth. The Earth's population continues to grow at a rate of 1.05\%~\cite{BASAK2020100335}. 

\begin{figure}
	\centering
	\includegraphics[width=0.7\linewidth]{../../../../../../../../Desktop/WorldPopulation}
	\caption[World Population from 1700's to Present Day]{World Population: Past, Present, and Future. The graph displays the population increase from the 1700's to present day.}~\cite{PopulationClock202007}
	\label{fig:worldpopulation}
\end{figure}

The planet’s resources are expected to cap at nearly 8 billion people. The Earth currently inhabits 7.8 billion as of July 2020. Earth’s estimated timeline to maintain complex life is anticipated to cease to exist as early as 1.5 billion years. Apart from discovering life at a habitable planet, space exploration allows humankind to gain a greater understanding of the cosmos and potentially answer the question of, “Is there life beyond Earth?”~\cite{ScienceMissionDirectorate2020} and more importantly, could we catapult the human race to this newfound habitable planet.

As mentioned earlier, the good news is that we have mounds of data that have been collected over the last centuries. What is lacking is a good way to determine patterns in all of the data with computational cycles of the human brain. NASA has collected Kepler data that is a repository of the type of information needed to answer some of the questions of  the human race. Unfortunately, as with most big data systems, the repository is data rich, but information poor. It is. becoming increasingly difficult to filter the data to meaningful levels. The data contains information on image sources, light measures, and gravity among other things. In conjunction with the volume of data is the unknown requirements that confirm if a planet is habitable, analyzing the parameters pouring in to confirm a planet, and understanding the cosmic web that connects the galaxies which hold potential habitable planets. 

This project's goal is aimed at answering the following problem: Based on the myriad data available about space exploration, is it possible to provide an estimate of the number of other exoplanets in the Universe that are able to maintain life as we know it? In addition, could a repeatable hosted data center be built that allows for the community to consume and build algorithms for this specific purpose; easing the issue of data filtering? The main contributions to this paper are mentioned  in the references section which supports the understanding of the science behind this data. In the early stages of the paper, the team will discuss the motivation behind the problems statement and then continue on into the methodologies considered. More focus will be aimed at the hosting and replication of the data using Amazon Web Services (AWS) than the actual algorithms used to determine habitual planets. Nonetheless, the team will demonstrate a few algorithm techniques to prove out the hosting platform and its feasibility. The team will then explore methodologies in deep learning and machine learning that are widely used in the application of exoplanet scientific research. This will entail creating a working model to detect exoplanet feasibility based on configurable parameters set by the scientists and showcase the platform of data that will be served against those models. The team will then move on to the various milestones of the project which are the ability to collect the data, store the data in the Cloud (AWS), configure ETL processes that can be consumed, and then deploy a few machine learning models that will consume the data store.

\section{Motivation}

Based on algorithms and programs provided in research, it is possible to reign in the available data and provide a predictable and repeatable method to determine whether exoplanets in the Universe are habitable. Having an affinity of space, data science, and the hosting of large data sets provides the background for such experiments to thrive. As mentioned in the Introduction, the Astronomy data community find themselves data rich, information poor. While mounds of data exists, there is so much that even filtering the data is problematic. The best solution would be to find ways to federate the data to specific fields of studies. If a scientist is trying to determine where the next Super Nova will occur, they need specific pieces of data. Another scientist looking for the best environments for nebulae will look at another set of data. The mission of this project is to find what data elements would be best suited for helping scientist determine exoplanets and find a predictable and repeatable method to host and consume that data. 

The only race on Earth is the human race, we must band together, using all technologies available to help sustain the human race. Earth is on borrowed time. Population growth and natural resource consumption will end tragically for Earth at some point in the future. Humans must invest in ways to either minimize natural resource consumption or look for other parts of the Universe that we could colonize. Armageddon comes in many forms: climate breakdown, asteroid strike, zombie apocalypse, and so on. There are scientist that choose to use the Copernican Principle with statistics to figure out how long anything will last. We could use this to determine within a 95\% confidence how long humans will be around. Right now, that number is somewhere between 5,130 to 7.8 million years if you assume that humans have been around for about 200,000 years~\cite{Koehrsen2018}. This is in close correlation with the mean duration of a mammal species; which is about 2million years. 

\begin{figure}
	\centering
	\includegraphics[width=0.7\linewidth]{"../../../../../../../../Desktop/Screen Shot 2021-02-28 at 9.00.34 AM"}
	\caption[The Copernican Lifetime Equation]{The Copernican Lifetime Equation}~\cite{Koehrsen2018}
	\label{fig:screen-shot-2021-02-28-at-9}
\end{figure}

While the motivation might not be there for our current generation; technology evolves as stepping stones from generation to generation. The need to support scientists in taking the next big steps of finding exoplanets is now. The technology is ripe, the desires are there, all that is needed are tools to move us in the right direction. The next question in generations to come might be, "how do we get to said exoplanet", but technology isn't really at a point to answer that question yet. The first step is to equip scientists with data techniques that allow them to pin point areas that could provide colonization options. The sooner we do this, the sooner we can move on to the next stepping stone of technology needs.

\section{Methodology}

Lorem ipsum dolor sit amet, consectetur adipiscing elit, sed do eiusmod tempor incididunt ut labore et dolore magna aliqua. Ut enim ad minim veniam, quis nostrud exercitation ullamco laboris nisi ut aliquip ex ea commodo consequat. Duis aute irure dolor in reprehenderit in voluptate velit esse cillum dolore eu fugiat nulla pariatur. Excepteur sint occaecat cupidatat non proident, sunt in culpa qui officia deserunt mollit anim id est laborum. Lorem ipsum dolor sit amet, consectetur adipiscing elit, sed do eiusmod tempor incididunt ut labore et dolore magna aliqua~\cite{PopulationClock202007}. Ut enim ad minim veniam, quis nostrud exercitation ullamco laboris nisi ut aliquip ex ea commodo consequat. Duis aute irure dolor in reprehenderit in voluptate velit esse cillum dolore eu fugiat nulla pariatur. Excepteur sint occaecat cupidatat non proident, sunt in culpa qui officia deserunt mollit anim id est laborum.

\section{Milstones-Data Collection}

Lorem ipsum dolor sit amet, consectetur adipiscing elit, sed do eiusmod tempor incididunt ut labore et dolore magna aliqua. Ut enim ad minim veniam, quis nostrud exercitation ullamco laboris nisi ut aliquip ex ea commodo consequat. Duis aute irure dolor in reprehenderit in voluptate velit esse cillum dolore eu fugiat nulla pariatur. Excepteur sint occaecat cupidatat non proident, sunt in culpa qui officia deserunt mollit anim id est laborum. Lorem ipsum dolor sit amet, consectetur adipiscing elit, sed do eiusmod tempor incididunt ut labore et dolore magna aliqua. Ut enim ad minim veniam, quis nostrud exercitation ullamco laboris nisi ut aliquip ex ea commodo consequat. Duis aute irure dolor in reprehenderit in voluptate velit esse cillum dolore eu fugiat nulla pariatur~\cite{PopulationEnvironment2017}. Excepteur sint occaecat cupidatat non proident, sunt in culpa qui officia deserunt mollit anim id est laborum. Lorem ipsum dolor sit amet, consectetur adipiscing elit, sed do eiusmod tempor incididunt ut labore et dolore magna aliqua. Ut enim ad minim veniam, quis nostrud exercitation ullamco laboris nisi ut aliquip ex ea commodo consequat. Duis aute irure dolor in reprehenderit in voluptate velit esse cillum dolore eu fugiat nulla pariatur. Excepteur sint occaecat cupidatat non proident, sunt in culpa qui officia deserunt mollit anim id est laborum~\cite{Cite6}.

\section{Milestones-Data Storage in the Cloud}

Lorem ipsum dolor sit amet, consectetur adipiscing elit, sed do eiusmod tempor incididunt ut labore et dolore magna aliqua. Ut enim ad minim veniam, quis nostrud exercitation ullamco laboris nisi ut aliquip ex ea commodo consequat. Duis aute irure dolor in reprehenderit in voluptate velit esse cillum dolore eu fugiat nulla pariatur. Excepteur sint occaecat cupidatat non proident, sunt in culpa qui officia deserunt mollit anim id est laborum. Lorem ipsum dolor sit amet, consectetur adipiscing elit, sed do eiusmod tempor incididunt ut labore et dolore magna aliqua~\cite{Cite7}. Ut enim ad minim veniam, quis nostrud exercitation ullamco laboris nisi ut aliquip ex ea commodo consequat. Duis aute irure dolor in reprehenderit in voluptate velit esse cillum dolore eu fugiat nulla pariatur. Excepteur sint occaecat cupidatat non proident, sunt in culpa qui officia deserunt mollit anim id est laborum~\cite{Cite8}.

\begin{itemize}
	\item Item 1
	\item Item 2
	\item Item 3
\end{itemize}

Lorem ipsum dolor sit amet, consectetur adipiscing elit, sed do eiusmod tempor incididunt ut labore et dolore magna aliqua. Ut enim ad minim veniam, quis nostrud exercitation ullamco laboris nisi ut aliquip ex ea commodo consequat. Duis aute irure dolor in reprehenderit in voluptate velit esse cillum dolore eu fugiat nulla pariatur~\cite{Cite9}. Excepteur sint occaecat cupidatat non proident, sunt in culpa qui officia deserunt mollit anim id est laborum. Lorem ipsum dolor sit amet, consectetur adipiscing elit, sed do eiusmod tempor incididunt ut labore et dolore magna aliqua. Ut enim ad minim veniam, quis nostrud exercitation ullamco laboris nisi ut aliquip ex ea commodo consequat. Duis aute irure dolor in reprehenderit in voluptate velit esse cillum dolore eu fugiat nulla pariatur. Excepteur sint occaecat cupidatat non proident, sunt in culpa qui officia deserunt mollit anim id est laborum~\cite{Cite10}.

Lorum ipsum and another table:

\begin{center}
\begin{table}[ht]
	\caption{Deanonymize Classification}
	\centering
	\begin{tabular}{l r r r r}
		\hline\hline
		\ \ & client  ACK & server data & server ACK & server data \\\\ [0.5ex] 
		\hline
		detection rate&96\%&94\%&96\%&94\% \\
		false negative&4\%&6\%&4\%&6\% \\
		false positive&0\%&0\%&0\% &0\% \\ [1ex]
		\hline
	\end{tabular}
	\label{table:nonlin}
\end{table}
\end{center} 

Lorem ipsum dolor sit amet~\cite{Cite11}, consectetur adipiscing elit, sed do eiusmod tempor incididunt ut labore et dolore magna aliqua. Ut enim ad minim veniam, quis nostrud exercitation ullamco laboris nisi ut aliquip ex ea commodo consequat. Duis aute irure dolor in reprehenderit in voluptate velit esse cillum dolore eu fugiat nulla pariatur. Excepteur sint occaecat cupidatat non proident, sunt in culpa qui officia deserunt mollit anim id est laborum. Lorem ipsum dolor sit amet, consectetur adipiscing elit, sed do eiusmod tempor incididunt ut labore et dolore magna aliqua. Ut enim ad minim veniam, quis nostrud exercitation ullamco laboris nisi ut aliquip ex ea commodo consequat. Duis aute irure dolor in reprehenderit in voluptate velit esse cillum dolore eu fugiat nulla pariatur. Excepteur sint occaecat cupidatat non proident, sunt in culpa qui officia deserunt mollit anim id est laborum.

\section{Milestones-ETL (Extract, Transform, and Load)}

Lorem ipsum dolor sit amet, consectetur adipiscing elit~\cite{Cite12}, sed do eiusmod tempor incididunt ut labore et dolore magna aliqua. Ut enim ad minim veniam, quis nostrud exercitation ullamco laboris nisi ut aliquip ex ea commodo consequat. Duis aute irure dolor in reprehenderit in voluptate velit esse cillum dolore eu fugiat nulla pariatur. Excepteur sint occaecat cupidatat non proident, sunt in culpa qui officia deserunt mollit anim id est laborum:

\begin{enumerate}
	\item Client: a client of the Tor network is targeted to identify it.
	\item Server: the Tor onion (hidden) service is targeted to reveal its identity or to weaken it.
	\item Network: the broader Tor network itself is targeted, usually via multiple malicious Tor nodes.
\end{enumerate}

Lorem ipsum dolor sit amet, consectetur adipiscing elit~\cite{Cite13}, sed do eiusmod tempor incididunt ut labore et dolore magna aliqua. Ut enim ad minim veniam, quis nostrud exercitation ullamco laboris nisi ut aliquip ex ea commodo consequat. Duis aute irure dolor in reprehenderit in voluptate velit esse cillum dolore eu fugiat nulla pariatur. Excepteur sint occaecat cupidatat non proident, sunt in culpa qui officia deserunt mollit anim id est laborum. Lorem ipsum dolor sit amet, consectetur adipiscing elit, sed do eiusmod tempor incididunt ut labore et dolore magna aliqua. Ut enim ad minim veniam, quis nostrud exercitation ullamco laboris nisi ut aliquip ex ea commodo consequat. Duis aute irure dolor in reprehenderit in voluptate velit esse cillum dolore eu fugiat nulla pariatur. Excepteur sint occaecat cupidatat non proident, sunt in culpa qui officia deserunt mollit anim id est laborum. Lorem ipsum dolor sit amet, consectetur adipiscing elit, sed do eiusmod tempor incididunt ut labore et dolore magna aliqua. Ut enim ad minim veniam, quis nostrud exercitation ullamco laboris nisi ut aliquip ex ea commodo consequat. Duis aute irure dolor in reprehenderit in voluptate velit esse cillum dolore eu fugiat nulla pariatur. Excepteur sint occaecat cupidatat non proident, sunt in culpa qui officia deserunt mollit anim id est laborum.

\section{Milstones-Machine Learning Models}

Lorem ipsum dolor sit amet, consectetur adipiscing elit~\cite{Cite14}, sed do eiusmod tempor incididunt ut labore et dolore magna aliqua. Ut enim ad minim veniam, quis nostrud exercitation ullamco laboris nisi ut aliquip ex ea commodo consequat. Duis aute irure dolor in reprehenderit in voluptate velit esse cillum dolore eu fugiat nulla pariatur. Excepteur sint occaecat cupidatat non proident, sunt in culpa qui officia deserunt mollit anim id est laborum. Lorem ipsum dolor sit amet, consectetur adipiscing elit, sed do eiusmod tempor incididunt ut labore et dolore magna aliqua. Ut enim ad minim veniam, quis nostrud exercitation ullamco~\cite{Cite15} laboris nisi ut aliquip ex ea commodo consequat. Duis aute irure dolor in reprehenderit in voluptate velit esse cillum dolore eu fugiat nulla pariatur. Excepteur sint occaecat cupidatat non proident, sunt in culpa qui officia deserunt mollit anim id est laborum.

\begin{center}
\begin{table}[ht]
	\caption{Traffic Analysis Through Flow Correlation}
	\centering
	\begin{tabular}{l r r}
		\hline\hline
		\ \ & In-Lab & Tor Relay \\\\ [0.5ex] 
		\hline
		De-anonymized&100\%&81.4\% \\
		false negative&0\%&12.2\% \\
		false positive&0\%&6.4\%\\ [1ex]
		\hline
	\end{tabular}
	\label{table:nonlin}
\end{table}
\end{center} 

Lorem ipsum dolor sit amet, consectetur adipiscing elit, sed do eiusmod tempor incididunt ut labore et dolore magna aliqua. Ut enim ad minim veniam, quis nostrud exercitation ullamco laboris nisi ut aliquip ex ea commodo consequat. Duis aute irure dolor in reprehenderit in voluptate velit esse cillum dolore eu fugiat nulla pariatur. Excepteur sint occaecat cupidatat non proident, sunt in culpa qui officia deserunt mollit anim id est laborum.

\section{CONCLUSIONS}

Lorem ipsum dolor sit amet, consectetur adipiscing elit, sed do eiusmod tempor incididunt ut labore et dolore magna aliqua. Ut enim ad minim veniam, quis nostrud exercitation ullamco laboris nisi ut aliquip ex ea commodo consequat. Duis aute irure dolor in reprehenderit in voluptate velit esse cillum dolore eu fugiat nulla pariatur. Excepteur sint occaecat cupidatat non proident, sunt in culpa qui officia deserunt mollit anim id est laborum.~\cite{Cite16}. Lorem ipsum dolor sit amet, consectetur adipiscing elit, sed do eiusmod tempor incididunt ut labore et dolore magna aliqua. Ut enim ad minim veniam, quis nostrud exercitation ullamco laboris nisi ut aliquip ex ea commodo consequat. Duis aute irure dolor in reprehenderit in voluptate velit esse cillum dolore eu fugiat nulla pariatur. Excepteur sint occaecat cupidatat non proident, sunt in culpa qui officia deserunt mollit anim id est laborum. Lorem ipsum dolor sit amet, consectetur adipiscing elit, sed do eiusmod tempor incididunt ut labore et dolore magna aliqua. Ut enim ad minim veniam, quis nostrud exercitation ullamco laboris nisi ut aliquip ex ea commodo consequat. Duis aute irure dolor in reprehenderit in voluptate velit esse cillum dolore eu fugiat nulla pariatur. Excepteur sint occaecat cupidatat non proident, sunt in culpa qui officia deserunt mollit anim id est laborum. Lorem ipsum dolor sit amet, consectetur adipiscing elit, sed do eiusmod tempor incididunt ut labore et dolore magna aliqua. Ut enim ad minim veniam, quis nostrud exercitation ullamco laboris nisi ut aliquip ex ea commodo consequat. Duis aute irure dolor in reprehenderit in voluptate velit esse cillum dolore eu fugiat nulla pariatur. Excepteur sint occaecat cupidatat non proident, sunt in culpa qui officia deserunt mollit anim id est laborum.

\addtolength{\textheight}{-12cm}   % This command serves to balance the column lengths
                                  % on the last page of the document manually. It shortens
                                  % the textheight of the last page by a suitable amount.
                                  % This command does not take effect until the next page
                                  % so it should come on the page before the last. Make
                                  % sure that you do not shorten the textheight too much.

%%%%%%%%%%%%%%%%%%%%%%%%%%%%%%%%%%%%%%%%%%%%%%%%%%%%%%%%%%%%%%%%%%%%%%%%%%%%%%%%

\bibliography{references}{}
\bibliographystyle{IEEEtran}
\end{document}